\documentclass[11pt]{ctexart}
\usepackage[margin=2cm,a4paper]{geometry}

\setmainfont{Caladea}

%% 也可以选用其它字库:
% \setCJKmainfont[%
%   ItalicFont=AR PL KaitiM GB,
%   BoldFont=Noto Sans CJK SC,
% ]{AR PL SungtiL GB}
% \setCJKsansfont{Noto Sans CJK SC}
% \renewcommand{\kaishu}{\CJKfontspec{AR PL KaitiM GB}}


\usepackage{minted}

\usepackage[breaklinks]{hyperref}

\title{在Overleaf平台上使用C\TeX}
\author{林莲枝}

\begin{document}
\maketitle

繁體中文可以嗎?

Overleaf 伺服器上安装的字体都是开源授权的,因此很多C\TeX{}模板设定的默认字体都不能使用(都是微软视窗或Adobe字体)。导致有些朋友上载了自己的C\TeX{}文件却不能编译,抓狂不已。

其实最重要的是,不要用 \verb|fontset=windows|!伺服器上没有 Windows 字体,真的有需要的话,请自行上传 .ttf 文件。

其实无需再加任何 \verb|fontset| 选项,系统会在 XeLaTeX 或 LuaLaTeX 下自动调用 Fandol,编译、显示都没有问题。调用效果如下:

\begin{itemize}
\item 一般字体(\verb|\rmfamily|)为〖宋体〗。
\item 需要强调时,Fandol \verb|\textbf| 用的是〖\textbf{加黑宋体}〗。
\item \verb|\sffamily| 用的是 〖\textsf{黑体}〗。
\item 中文字体是没有斜体的,因此 \verb|\emph|和 \verb|\textit| 都是〖\textit{楷体}〗。

\item 单距字体(\verb|\ttfamily|)很多人爱用\texttt{〖仿宋〗}。

\end{itemize}


想调用其它字库的话,伺服器上现有的中文字库可参考这个列表:\url{https://www.overleaf.com/help/193#!CJK}。

也可以自行上载 TTF/OTF 档案,直接用档案名来,如:\mintinline{LaTeX}{\setmainfont{shuti.otf}}。目前伺服器上没有幼圆、隶书字库可供C\TeX{}直接使用,抱歉了。


繁体的话,先用\verb|nofonts|参数,再用\verb|fontspec|方法来配置字库。可以考虑 cwTeXKai, cwTeXMing, cwTeXHeiBold, cwTeXYen。比如:

\begin{minted}{LaTeX}
\documentclass[nofonts]{ctexart}
\setCJKmainfont[
  BoldFont={cwTeXHeiBold},
  ItalicFont={cwTeXKai}]
{cwTeXMing}
\setCJKsansfont{cwTeXHei}
\setCJKmonofont{cwTeXYen}
\end{minted}

\end{document}