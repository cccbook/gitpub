\documentclass{article}

\usepackage{arxiv}

\usepackage[utf8]{inputenc} % allow utf-8 input
\usepackage[T1]{fontenc}    % use 8-bit T1 fonts
\usepackage{hyperref}       % hyperlinks
\usepackage{url}            % simple URL typesetting
\usepackage{booktabs}       % professional-quality tables
\usepackage{amsfonts}       % blackboard math symbols
\usepackage{nicefrac}       % compact symbols for 1/2, etc.
\usepackage{microtype}      % microtypography
\usepackage{lipsum}    % 自動產生亂文
\usepackage{graphicx}  % 使用圖片
\usepackage{listings}   % 使用程式碼嵌入
\usepackage{fontspec}   %加這個就可以設定字體
\usepackage{xeCJK}       %讓中英文字體分開設置
\setCJKmainfont{細明體} % SimSun, 標楷體, 設定中文為系統上的字型,而英文不去更動,使用原TeX字型
\XeTeXlinebreaklocale "zh"             %這兩行一定要加,中文才能自動換行
\XeTeXlinebreakskip = 0pt plus 1pt     %這兩行一定要加,中文才能自動換行
\renewcommand{\baselinestretch}{1.5} % 行與行之間的間距加大 1.5 倍


\title{從爬山演算法開始學習人工智慧的優化算法}


\author{
  陳鍾誠\thanks{使用註腳來進一步說明}\\
  金門大學資訊工程系\\
  \texttt{ccc@nqu.edu.tw} \\
  %% examples of more authors
}

\begin{document}
\maketitle

\begin{abstract}
我們使用爬山演算法,結合編譯器技術,創造了一個自動求的符號的解方程式套件 eq6.js,雖然並非所有方程式都能求得符號解,但是對《線性方程組、多項式與常係數微分方程式》而言,通常可以求得正確解答,而對其他更複雜的微分方程或偏微分方程,則無法保證能得到正確解答。
\end{abstract}


% keywords can be removed
\keywords{方程式求解\and 符號微分\and 人工智慧}

\input{body}


\bibliographystyle{unsrt}  
%\bibliography{references}  %%% Remove comment to use the external .bib file (using bibtex).
%%% and comment out the ``thebibliography'' section.


%%% Comment out this section when you \bibliography{references} is enabled.
\begin{thebibliography}{1}

\bibitem{}
陳鍾誠
\newblock 爬山演算法
\newblock (2017) github pages, e103.

\bibitem{10.7717/peerj-cs.103}
Meurer, Aaron and Smith, Christopher P. and Paprocki, Mateusz
\newblock SymPy: symbolic computing in Python,
\newblock (2017) SymPy: symbolic computing in Python. PeerJ Computer Science 3:e103.

\bibitem{kour2014real}
George Kour and Raid Saabne.
\newblock Real-time segmentation of on-line handwritten arabic script.
\newblock In {\em Frontiers in Handwriting Recognition (ICFHR), 2014 14th
  International Conference on}, pages 417--422. IEEE, 2014.

\bibitem{kour2014fast}
George Kour and Raid Saabne.
\newblock Fast classification of handwritten on-line arabic characters.
\newblock In {\em Soft Computing and Pattern Recognition (SoCPaR), 2014 6th
  International Conference of}, pages 312--318. IEEE, 2014.

\bibitem{hadash2018estimate}
Guy Hadash, Einat Kermany, Boaz Carmeli, Ofer Lavi, George Kour, and Alon
  Jacovi.
\newblock Estimate and replace: A novel approach to integrating deep neural
  networks with existing applications.
\newblock {\em arXiv preprint arXiv:1804.09028}, 2018.

\end{thebibliography}


\end{document}
